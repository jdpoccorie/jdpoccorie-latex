\documentclass{article}
\usepackage[utf8]{inputenc}
\usepackage[spanish]{babel}
\usepackage{amsmath}
\usepackage{amsfonts}
\usepackage{amssymb}
\usepackage{lipsum}
\usepackage{xstring} % Paquete necesario para cargar condicionales en newcommand
\usepackage{graphicx} % Paquete necesario que nos permite trabajar con imágenes
\usepackage{vmargin} % Definir los margenes del documento
\usepackage{wrapfig}
\usepackage{subcaption}
\usepackage{slashbox} % Paquete que nos permitira dividir una celda con diagonal revisar si en verdad contamos con el paquete

% Paquetes que usaremos para eñ color en nuestras tablas
\usepackage{colortbl}
\usepackage[table]{xcolor} %podemos usar este paquete en el archivo paletaColores

% Cargar los paquetes para la combinacion de columnas y de filas
\usepackage{multicol, multirow}


\setpapersize{A4}
\setmargins{2.5cm} % margen izquierdo
{1.5cm} % margen superior
{16.5cm} % ancho del area de impresion
{23.42cm} % longitud del area de impresion
{0pt} % espacio para el encabezado
{5mm} % espacio entre el encabezado y el texto
{0pt} % espacio para el pie de pagina
{5mm} % espacio entre el texto y el pie de pagina

% indicamos la direccion donde tenemos nuestras imagenes
\graphicspath{{./images/}}

% Importar paleta de colores
%archivo paletaColores.tex
\usepackage{xcolor} % Importar el paquete xcolor

% Definir mi paleta de colores
\definecolor{myGreen}{HTML}{36A736}
\definecolor{myBlue}{HTML}{02528F}
\definecolor{blue254}{HTML}{02528F}
\definecolor{myOrange}{HTML}{FF4312}

\title{Ejemplo de uso de Imágenes y Texto en \LaTeX}
\author{Juan Diego Poccori Escalante}
\date{\today}

\begin{document}
    \maketitle
    \renewcommand{\contentsname}{Tabla de contenido}
    \renewcommand{\listfigurename}{Lista de Figuras}
    \renewcommand{\figurename}{Fig.}
    \renewcommand{\listtablename}{Lista de tablas}
    \tableofcontents
    \listoffigures
    \listoftables

    \section{Combinar celdas}
    \begin{table}[ht]
        \centering
        \begin{tabular}{>{\cellcolor{myBlue!75}}c|c|c|c|}
            \hline
            \rowcolor{myBlue!75}
            & \multicolumn{3}{c|}{\textcolor{white}{Tolerancia Resistiva($\pm$)}} \\
            \rowcolor{myBlue!75}
            & 40\% & 20\% & 10\% \\

            & \multirow{4}{*}{100} & \multirow{2}{*}{100} & 100 \\
            \cline{4-4} % Dibuja la linea horizontal indicando de que columna a que columna se dibijara
            & & & 91 \\
            \cline{3-4}
            & & \multirow{2}{*}{82} & 82 \\
            \cline{4-4}
            & & & 75 \\
            \cline{2-4}

            & \multirow{4}{*}{78} & \multirow{2}{*}{68} & 68 \\
            \cline{4-4}
            & & & 81 \\
            \cline{3-4}
            & & \multirow{2}{*}{92} & 72 \\
            \cline{4-4}
            & & & 85 \\
            \cline{2-4}

            & \multirow{4}{*}{88} & \multirow{2}{*}{68} & 68 \\
            \cline{4-4}
            & & & 71 \\
            \cline{3-4}
            & & \multirow{2}{*}{102} & 62 \\
            \cline{4-4}
            & & & 95 \\
            \cline{2-4}
            \multirow{-14}{*}{\rotatebox[origin=c]{90}{\textcolor{white}{Valores de resitencia estándar}}} & 10 & 10 & 10 \\
            \hline

        \end{tabular}
        
    \end{table}

   
\end{document}
